% Options for packages loaded elsewhere
% Options for packages loaded elsewhere
\PassOptionsToPackage{unicode}{hyperref}
\PassOptionsToPackage{hyphens}{url}
%
\documentclass[
  ignorenonframetext,
  aspectratio=169,
]{beamer}
\newif\ifbibliography
\usepackage{pgfpages}
\setbeamertemplate{caption}[numbered]
\setbeamertemplate{caption label separator}{: }
\setbeamercolor{caption name}{fg=normal text.fg}
\beamertemplatenavigationsymbolshorizontal
% remove section numbering
\setbeamertemplate{part page}{
  \centering
  \begin{beamercolorbox}[sep=16pt,center]{part title}
    \usebeamerfont{part title}\insertpart\par
  \end{beamercolorbox}
}
\setbeamertemplate{section page}{
  \centering
  \begin{beamercolorbox}[sep=12pt,center]{section title}
    \usebeamerfont{section title}\insertsection\par
  \end{beamercolorbox}
}
\setbeamertemplate{subsection page}{
  \centering
  \begin{beamercolorbox}[sep=8pt,center]{subsection title}
    \usebeamerfont{subsection title}\insertsubsection\par
  \end{beamercolorbox}
}
% Prevent slide breaks in the middle of a paragraph
\widowpenalties 1 10000
\raggedbottom
\AtBeginPart{
  \frame{\partpage}
}
\AtBeginSection{
  \ifbibliography
  \else
    \frame{\sectionpage}
  \fi
}
\AtBeginSubsection{
  \frame{\subsectionpage}
}
\usepackage{iftex}
\ifPDFTeX
  \usepackage[T1]{fontenc}
  \usepackage[utf8]{inputenc}
  \usepackage{textcomp} % provide euro and other symbols
\else % if luatex or xetex
  \usepackage{unicode-math} % this also loads fontspec
  \defaultfontfeatures{Scale=MatchLowercase}
  \defaultfontfeatures[\rmfamily]{Ligatures=TeX,Scale=1}
\fi
\usepackage{lmodern}

\usetheme[]{metropolis}
\usecolortheme[]{dolphin}
\usefonttheme[]{professionalfonts}
\ifPDFTeX\else
  % xetex/luatex font selection
\fi
% Use upquote if available, for straight quotes in verbatim environments
\IfFileExists{upquote.sty}{\usepackage{upquote}}{}
\IfFileExists{microtype.sty}{% use microtype if available
  \usepackage[]{microtype}
  \UseMicrotypeSet[protrusion]{basicmath} % disable protrusion for tt fonts
}{}
\makeatletter
\@ifundefined{KOMAClassName}{% if non-KOMA class
  \IfFileExists{parskip.sty}{%
    \usepackage{parskip}
  }{% else
    \setlength{\parindent}{0pt}
    \setlength{\parskip}{6pt plus 2pt minus 1pt}}
}{% if KOMA class
  \KOMAoptions{parskip=half}}
\makeatother


\usepackage{longtable,booktabs,array}
\usepackage{calc} % for calculating minipage widths
\usepackage{caption}
% Make caption package work with longtable
\makeatletter
\def\fnum@table{\tablename~\thetable}
\makeatother
\usepackage{graphicx}
\makeatletter
\newsavebox\pandoc@box
\newcommand*\pandocbounded[1]{% scales image to fit in text height/width
  \sbox\pandoc@box{#1}%
  \Gscale@div\@tempa{\textheight}{\dimexpr\ht\pandoc@box+\dp\pandoc@box\relax}%
  \Gscale@div\@tempb{\linewidth}{\wd\pandoc@box}%
  \ifdim\@tempb\p@<\@tempa\p@\let\@tempa\@tempb\fi% select the smaller of both
  \ifdim\@tempa\p@<\p@\scalebox{\@tempa}{\usebox\pandoc@box}%
  \else\usebox{\pandoc@box}%
  \fi%
}
% Set default figure placement to htbp
\def\fps@figure{htbp}
\makeatother





\setlength{\emergencystretch}{3em} % prevent overfull lines

\providecommand{\tightlist}{%
  \setlength{\itemsep}{0pt}\setlength{\parskip}{0pt}}



 


\makeatletter
\@ifpackageloaded{caption}{}{\usepackage{caption}}
\AtBeginDocument{%
\ifdefined\contentsname
  \renewcommand*\contentsname{Table of contents}
\else
  \newcommand\contentsname{Table of contents}
\fi
\ifdefined\listfigurename
  \renewcommand*\listfigurename{List of Figures}
\else
  \newcommand\listfigurename{List of Figures}
\fi
\ifdefined\listtablename
  \renewcommand*\listtablename{List of Tables}
\else
  \newcommand\listtablename{List of Tables}
\fi
\ifdefined\figurename
  \renewcommand*\figurename{Figure}
\else
  \newcommand\figurename{Figure}
\fi
\ifdefined\tablename
  \renewcommand*\tablename{Table}
\else
  \newcommand\tablename{Table}
\fi
}
\@ifpackageloaded{float}{}{\usepackage{float}}
\floatstyle{ruled}
\@ifundefined{c@chapter}{\newfloat{codelisting}{h}{lop}}{\newfloat{codelisting}{h}{lop}[chapter]}
\floatname{codelisting}{Listing}
\newcommand*\listoflistings{\listof{codelisting}{List of Listings}}
\makeatother
\makeatletter
\makeatother
\makeatletter
\@ifpackageloaded{caption}{}{\usepackage{caption}}
\@ifpackageloaded{subcaption}{}{\usepackage{subcaption}}
\makeatother
\makeatletter
\@ifpackageloaded{fontawesome5}{}{\usepackage{fontawesome5}}
\makeatother

\usepackage{bookmark}
\IfFileExists{xurl.sty}{\usepackage{xurl}}{} % add URL line breaks if available
\urlstyle{same}
\hypersetup{
  pdftitle={Homework Assignments},
  hidelinks,
  pdfcreator={LaTeX via pandoc}}


\title{Homework Assignments}
\author{}
\date{}

\begin{document}
\frame{\titlepage}


\begin{frame}
This page contains information about and a schedule of the homework
assignments for the semester.
\end{frame}

\begin{frame}[fragile]{General Information}
\phantomsection\label{general-information}
\begin{itemize}
\tightlist
\item
  While the instructions for each assignment are available through the
  linked pages for quick and public access, \textbf{if you are in the
  class you must use the link provided in \href{http://us.edstem.org}{Ed
  Discussion} to accept the assignment}. This will ensure that:

  \begin{itemize}
  \tightlist
  \item
    You have compatible versions of all relevant packages provided in
    the environment;
  \item
    You have a GitHub repository that you can use to share your code.
  \end{itemize}
\item
  Submit assignments by 9:00pm Eastern Time on the due date on
  \href{https://gradescope.com}{Gradescope}.
\item
  Every student should submit their own PDF representing their own
  understanding of the material. If you worked with others, make sure to
  credit them.
\item
  More detailed
  \href{../policies/homework.qmd\#homework-logistics}{homework
  instructions and logistics} are available.
\item
  Submissions must be PDFs. Make sure that you tag the pages
  corresponding to each question; points will be deducted otherwise.
\item
  To convert the assignment notebook to PDF, you can use VS Code to
  render the notebook to HTML, and then use your browser to print to
  PDF. If you have set up LaTeX with VS Code, you can convert directly
  to a PDF.
\item
  As an alternative, when you commit and push a \texttt{.ipynb} file (a
  Jupyter Notebook) to your repository, we have set up a
  \href{https://docs.github.com/en/actions}{GitHub Action} to
  automatically render your notebook to a PDF, which you can then
  download and submit to Gradescope. However, sometimes GitHub actions
  can take a while or can stall out, so you'll need to monitor this and
  give yourself some time. If you've waited a while and your notebook
  isn't rendering, reach out on Ed and we can figure out what's going
  on. In the worst case (you push your changes close to the deadline),
  the timestamp of your commit will be evidence that you completed the
  assignment on time, even if we can't render the PDF until the next
  day.
\end{itemize}
\end{frame}

\begin{frame}{Grading}
\phantomsection\label{grading}
Make sure to look over the \href{../policies/rubric.qmd}{standard
homework rubric} and familiarize yourself with the
\href{../policies/homework.qmd}{homework} and
\href{../policies/grading.qmd}{grading policies}.
\end{frame}

\begin{frame}{Schedule}
\phantomsection\label{schedule}
\begin{longtable}[]{@{}
  >{\centering\arraybackslash}p{(\linewidth - 10\tabcolsep) * \real{0.0500}}
  >{\centering\arraybackslash}p{(\linewidth - 10\tabcolsep) * \real{0.1500}}
  >{\centering\arraybackslash}p{(\linewidth - 10\tabcolsep) * \real{0.0500}}
  >{\centering\arraybackslash}p{(\linewidth - 10\tabcolsep) * \real{0.0500}}
  >{\centering\arraybackslash}p{(\linewidth - 10\tabcolsep) * \real{0.1000}}
  >{\centering\arraybackslash}p{(\linewidth - 10\tabcolsep) * \real{0.1000}}@{}}
\toprule\noalign{}
\begin{minipage}[b]{\linewidth}\centering
Assignment
\end{minipage} & \begin{minipage}[b]{\linewidth}\centering
Topic
\end{minipage} & \begin{minipage}[b]{\linewidth}\centering
Instructions
\end{minipage} & \begin{minipage}[b]{\linewidth}\centering
Solutions
\end{minipage} & \begin{minipage}[b]{\linewidth}\centering
Assigned
\end{minipage} & \begin{minipage}[b]{\linewidth}\centering
Due
\end{minipage} \\
\midrule\noalign{}
\endhead
HW 1 & Intro and Exploratory Data Analysis &
\href{hw01/hw01.qmd}{\faIcon{pen-field}} & {\faIcon{key-skeleton}} &
Jan.~21, 2026 & Feb.~06, 2026 \\
HW 2 & Linear Models and DAGs & {\faIcon{pen-field}}
{\faIcon{key-skeleton}} & Feb.~09, 2026 & Feb.~20, 2026 & \\
HW 3 & Generalized Linear Models and Time Series & {\faIcon{pen-field}}
{\faIcon{key-skeleton}} & Feb.~23, 2026 & Mar.~06, 2026 & \\
HW 4 & Monte Carlo & {\faIcon{pen-field}} {\faIcon{key-skeleton}} &
Mar.~16, 2026 & Mar.~27, 2026 & \\
HW 5 & The Bootstrap and Multiple Imputation & {\faIcon{pen-field}}
{\faIcon{key-skeleton}} & Apr.~06, 2026 & Apr.~17, 2026 & \\
HW 6 & Model Scoring and Hypothesis Testing & {\faIcon{pen-field}}
{\faIcon{key-skeleton}} & Apr.~20, 2026 & May. 20, 2026 & \\
\bottomrule\noalign{}
\end{longtable}
\end{frame}




\end{document}
