% Options for packages loaded elsewhere
% Options for packages loaded elsewhere
\PassOptionsToPackage{unicode}{hyperref}
\PassOptionsToPackage{hyphens}{url}
%
\documentclass[
  ignorenonframetext,
  aspectratio=169,
]{beamer}
\newif\ifbibliography
\usepackage{pgfpages}
\setbeamertemplate{caption}[numbered]
\setbeamertemplate{caption label separator}{: }
\setbeamercolor{caption name}{fg=normal text.fg}
\beamertemplatenavigationsymbolshorizontal
% remove section numbering
\setbeamertemplate{part page}{
  \centering
  \begin{beamercolorbox}[sep=16pt,center]{part title}
    \usebeamerfont{part title}\insertpart\par
  \end{beamercolorbox}
}
\setbeamertemplate{section page}{
  \centering
  \begin{beamercolorbox}[sep=12pt,center]{section title}
    \usebeamerfont{section title}\insertsection\par
  \end{beamercolorbox}
}
\setbeamertemplate{subsection page}{
  \centering
  \begin{beamercolorbox}[sep=8pt,center]{subsection title}
    \usebeamerfont{subsection title}\insertsubsection\par
  \end{beamercolorbox}
}
% Prevent slide breaks in the middle of a paragraph
\widowpenalties 1 10000
\raggedbottom
\AtBeginPart{
  \frame{\partpage}
}
\AtBeginSection{
  \ifbibliography
  \else
    \frame{\sectionpage}
  \fi
}
\AtBeginSubsection{
  \frame{\subsectionpage}
}
\usepackage{iftex}
\ifPDFTeX
  \usepackage[T1]{fontenc}
  \usepackage[utf8]{inputenc}
  \usepackage{textcomp} % provide euro and other symbols
\else % if luatex or xetex
  \usepackage{unicode-math} % this also loads fontspec
  \defaultfontfeatures{Scale=MatchLowercase}
  \defaultfontfeatures[\rmfamily]{Ligatures=TeX,Scale=1}
\fi
\usepackage{lmodern}

\usetheme[]{metropolis}
\usecolortheme[]{dolphin}
\usefonttheme[]{professionalfonts}
\ifPDFTeX\else
  % xetex/luatex font selection
\fi
% Use upquote if available, for straight quotes in verbatim environments
\IfFileExists{upquote.sty}{\usepackage{upquote}}{}
\IfFileExists{microtype.sty}{% use microtype if available
  \usepackage[]{microtype}
  \UseMicrotypeSet[protrusion]{basicmath} % disable protrusion for tt fonts
}{}
\makeatletter
\@ifundefined{KOMAClassName}{% if non-KOMA class
  \IfFileExists{parskip.sty}{%
    \usepackage{parskip}
  }{% else
    \setlength{\parindent}{0pt}
    \setlength{\parskip}{6pt plus 2pt minus 1pt}}
}{% if KOMA class
  \KOMAoptions{parskip=half}}
\makeatother


\usepackage{longtable,booktabs,array}
\usepackage{calc} % for calculating minipage widths
\usepackage{caption}
% Make caption package work with longtable
\makeatletter
\def\fnum@table{\tablename~\thetable}
\makeatother
\usepackage{graphicx}
\makeatletter
\newsavebox\pandoc@box
\newcommand*\pandocbounded[1]{% scales image to fit in text height/width
  \sbox\pandoc@box{#1}%
  \Gscale@div\@tempa{\textheight}{\dimexpr\ht\pandoc@box+\dp\pandoc@box\relax}%
  \Gscale@div\@tempb{\linewidth}{\wd\pandoc@box}%
  \ifdim\@tempb\p@<\@tempa\p@\let\@tempa\@tempb\fi% select the smaller of both
  \ifdim\@tempa\p@<\p@\scalebox{\@tempa}{\usebox\pandoc@box}%
  \else\usebox{\pandoc@box}%
  \fi%
}
% Set default figure placement to htbp
\def\fps@figure{htbp}
\makeatother





\setlength{\emergencystretch}{3em} % prevent overfull lines

\providecommand{\tightlist}{%
  \setlength{\itemsep}{0pt}\setlength{\parskip}{0pt}}



 


\makeatletter
\@ifpackageloaded{caption}{}{\usepackage{caption}}
\AtBeginDocument{%
\ifdefined\contentsname
  \renewcommand*\contentsname{Table of contents}
\else
  \newcommand\contentsname{Table of contents}
\fi
\ifdefined\listfigurename
  \renewcommand*\listfigurename{List of Figures}
\else
  \newcommand\listfigurename{List of Figures}
\fi
\ifdefined\listtablename
  \renewcommand*\listtablename{List of Tables}
\else
  \newcommand\listtablename{List of Tables}
\fi
\ifdefined\figurename
  \renewcommand*\figurename{Figure}
\else
  \newcommand\figurename{Figure}
\fi
\ifdefined\tablename
  \renewcommand*\tablename{Table}
\else
  \newcommand\tablename{Table}
\fi
}
\@ifpackageloaded{float}{}{\usepackage{float}}
\floatstyle{ruled}
\@ifundefined{c@chapter}{\newfloat{codelisting}{h}{lop}}{\newfloat{codelisting}{h}{lop}[chapter]}
\floatname{codelisting}{Listing}
\newcommand*\listoflistings{\listof{codelisting}{List of Listings}}
\makeatother
\makeatletter
\makeatother
\makeatletter
\@ifpackageloaded{caption}{}{\usepackage{caption}}
\@ifpackageloaded{subcaption}{}{\usepackage{subcaption}}
\makeatother
\makeatletter
\@ifpackageloaded{fontawesome5}{}{\usepackage{fontawesome5}}
\makeatother

\usepackage{bookmark}
\IfFileExists{xurl.sty}{\usepackage{xurl}}{} % add URL line breaks if available
\urlstyle{same}
\hypersetup{
  pdftitle={Labs},
  hidelinks,
  pdfcreator={LaTeX via pandoc}}


\title{Labs}
\author{}
\date{}

\begin{document}
\frame{\titlepage}


\begin{frame}
This page contains information about and a schedule of the lab
assignments for the semester.
\end{frame}

\begin{frame}{General Information}
\phantomsection\label{general-information}
\begin{itemize}
\tightlist
\item
  You can download each lab notebook through the link below..
\item
  Submit assignments by 9:00pm Eastern Time on the due date on
  \href{https://gradescope.com}{Gradescope}.
\item
  Submissions must be PDFs. Make sure that you tag the pages
  corresponding to each question; points will be deducted otherwise.
\item
  To convert the assignment notebook to PDF, you can use VS Code to
  render the notebook to HTML, and then use your browser to print to
  PDF. If you have set up LaTeX with VS Code, you can convert directly
  to a PDF.
\end{itemize}
\end{frame}

\begin{frame}{Grading}
\phantomsection\label{grading}
Labs will be graded on a scale of 0-3:

\begin{itemize}
\tightlist
\item
  Missing (0/3): Lab solution is missing or is not responsive to the lab
  prompt(s);
\item
  Needs Improvement (1/3): Lab is largely incomplete or is missing key
  concepts;
\item
  Developing (2/3): Lab is mostly complete but may contain major
  conceptual or implementation errors;
\item
  Acceptable (3/3): Lab is mostly or fully complete without any major
  errors.
\end{itemize}
\end{frame}

\begin{frame}{Schedule}
\phantomsection\label{schedule}
\begin{longtable}[]{@{}
  >{\centering\arraybackslash}p{(\linewidth - 10\tabcolsep) * \real{0.0500}}
  >{\centering\arraybackslash}p{(\linewidth - 10\tabcolsep) * \real{0.1500}}
  >{\centering\arraybackslash}p{(\linewidth - 10\tabcolsep) * \real{0.0500}}
  >{\centering\arraybackslash}p{(\linewidth - 10\tabcolsep) * \real{0.1000}}
  >{\centering\arraybackslash}p{(\linewidth - 10\tabcolsep) * \real{0.1000}}
  >{\centering\arraybackslash}p{(\linewidth - 10\tabcolsep) * \real{0.1000}}@{}}
\toprule\noalign{}
\begin{minipage}[b]{\linewidth}\centering
Lab
\end{minipage} & \begin{minipage}[b]{\linewidth}\centering
Topic
\end{minipage} & \begin{minipage}[b]{\linewidth}\centering
Instructions
\end{minipage} & \begin{minipage}[b]{\linewidth}\centering
Class Date
\end{minipage} & \begin{minipage}[b]{\linewidth}\centering
Due
\end{minipage} & \begin{minipage}[b]{\linewidth}\centering
\end{minipage} \\
\midrule\noalign{}
\endhead
Lab 1 & Data Visualization & \href{lab01/lab01.qmd}{\faIcon{pen-field}}
& {\faIcon{key-skeleton}} & Jan.~30 & Feb.~2 \\
Lab 2 & Causal Reasoning & {\faIcon{pen-field}} & Feb.~13 & Feb.~16 & \\
Lab 3 & Generalized Linear Models & {\faIcon{pen-field}} & Feb.~20 &
Feb.~23 & \\
Lab 4 & Extreme Values & {\faIcon{pen-field}} & Mar.~6 & Mar.~9 & \\
Lab 5 & Monte Carlo & {\faIcon{pen-field}} & Mar.~20 & Mar.~23 & \\
Lab 6 & Missing Data & {\faIcon{pen-field}} & Apr.~10 & Apr.~13 & \\
Lab 7 & Model Comparison & {\faIcon{pen-field}} & Apr.~24 & Apr.~27 & \\
\bottomrule\noalign{}
\end{longtable}
\end{frame}




\end{document}
